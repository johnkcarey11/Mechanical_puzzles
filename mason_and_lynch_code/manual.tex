\documentstyle[12pt]{article}

\textheight 8.5in
\textwidth 6.8in
\topmargin -0.25in
\oddsidemargin -0.25in
\evensidemargin -0.25in

\begin{document}
\begin{centering}
{\bf{\LARGE The Stable Pushing Planner}} \\
\vspace*{0.2in}
Kevin M. Lynch \hspace{0.2in} Constantinos Nikou \hspace{0.2in} Matthew T. Mason\\
The Robotics Institute \\
Carnegie Mellon University \\
August 20, 1995 \\
\end{centering}

\section{Overview}

This document describes the code implementing the stable pushing planner
presented in ``Stable Pushing:  Mechanics, Controllability, and Planning,''
by Kevin M. Lynch and Matthew T. Mason.  This paper is to appear in the
International Journal of Robotics Research and is available on the
World Wide Web at 
{\verb+http://www.cs.cmu.edu/afs/cs/usr/lynch/research/kml-stable.html+}.

The code allows a user to automatically find pushing plans to move an
object among obstacles.  The strategy uses stable pushes, which keep
the pushed object fixed to the pusher as it moves.  The pusher is
assumed to be a flat edge.  The code consists of three major portions:

\begin{enumerate}
\item The object editor, which allows the user to enter a polygonal object
using a graphical interface.  The user must specify the center of mass of
the object, the coefficient of friction at the pushing contacts,
and the edges of the object that the robot can push.
The user may specify that the pusher act on the convex hull of the object.
The editor then automatically determines the set of
stable pushing directions for each edge, for use in the pushing planner.

\item The pushing world editor, which allows the user to specify pushing
problems for the objects created with the object editor.  This consists
of drawing the obstacles and specifying the start and goal positions.
The editor also invokes the planner, and if a solution is found, it 
displays and animates the path.

\item The push planner.  This is normally invoked by the world editor,
but it can also be used from the command line.  The planner also (if
desired) writes an Adept V+ file to execute the pushing path on an
Adept robot.
\end{enumerate}

The object and world editors are written in Tcl/Tk, and the planner is
written in C.  For best results, the editor should be run on a fast machine
with a color monitor and plenty of swap space.

\section{The object editor}

The object editor is a graphical interface that allows the user to
analyze objects for pushing.  The object editor is invoked from within the
world editor, described in the next section.  

The polygonal object is entered by mouse clicks.  The center of mass
(technically, its center of friction)
is assumed to be at (0, 0), and the user must specify
the coefficient of friction at the pushing contacts.  By clicking on
an edge of the object, the user marks (or unmarks) it as a possible
edge for pushing.  The user may also choose to use the pushing edges
defined by the convex hull of the object.

Once the user has entered the object, specified the friction coefficient,
and chosen a set of possible pushing edges, the set of stable pushing
directions for each edge can be calculated.  The planner actually uses
only the extreme velocity directions, as described in the paper.   
These velocity directions are displayed as $(v_x, v_y, \omega)$, where
$v_x$ is the $x$ velocity of the center of mass, $v_y$ is the $y$ velocity,
and $\omega$ is the angular velocity.  
When the user runs the analysis, lines will appear on the screen
that bound the sets of stable rotation centers.  These lines are described
in Figure~11 of the paper.  They are provided to aid the user's intuition
when checking the stable pushing directions for a single edge. 

Once a set of pushing edges has been chosen and the stable pushing
directions have been calculated, the user can save the object to the
file {\tt objects.tcl} read by the world editor.  The object is now
available for specifying pushing problems.  In practice, only a small
number of pushing edges should be used, as the time and space required
by the search increases with the number of available pushing motions.
It is desirable, however, that the set of pushing edges yield {\em
small-time local controllability\/}, so the object can be
maneuvered in tight spaces.  See the paper for details.

Some details to be aware of:
\begin{itemize}
\item There is no function to load a previously analyzed object into the
object editor, or for deleting an object from {\tt objects.tcl}.
The user can do this manually by editing the file.
\item The editor may be excessively slow if the grid spacing (under
{\tt Options}) is too small.  This is because the editor must draw crosses
at each grid point.
\end{itemize}

\subsection{Global variables}

Colors used in the object editor are defined in {\tt editcolors.tcl}.
The following global variables are defined in 
{\tt obeditglobals.tcl}.

\begin{description}
\item[SAFETY]  As described in Figure 11(b) of the paper, the lines 
a distance $r^2/p$ from the center of mass of the object should be
slightly more distant.  This is accounted for by multiplying 
$r^2/p$ by the constant SAFETY.  The default value is 1.0.
\item[FUZZY]  This constant is used for fuzzy math, to account
for loss of precision by floating point math.  For example, to 
test if a point lies on a line, we actually only require that it
lies very close to it. 
\item[FuzzyDist]  If two extremal stable rotation centers are within a distance
FuzzyDist of each other, they are assumed equivalent, and one of them 
is discarded.  
\item [MaxRCDist]  If a stable rotation center is greater than MaxRCDist
from the center of mass of the object, then it is essentially a translation,
and it is discarded.
\end{description}

\section{The world editor}

The world editor allows the user to draw obstacle fields, move the
start and goal position of the object, invoke the planner, display
the results of the planner, and print them to a postscript file.
The interface should be mostly self-explanatory.

The {\tt Options} button allows the user to change the size of the
pushing world and the resolution of the snap grid.  The user can
also specify the cost function for the planner.  The cost of the
plan is given by the cost of the action (pushing direction) changes,
the pushing edge changes, and the number of pushes.  The cost for
each of these must be integral, as described in the paper.  The user
may also specify the size of the goal region, the pushing step length,
and the size of the configuration space grid used to check for prior
occupancy.  (All angles are specified in degrees.)  We recommend
using the default goal size, step length, and grid size to ensure
``reasonable'' run times for the planner.

Some things to know:
\begin{itemize}
\item In most text entries (except the Load, Save, and PS file windows) 
pressing Return will not secure the entry, but will write a string
similar to ``\verb+\+0xd''.  This string {\em will\/} be included in the
entry, so it is possible to enter, for example, a grid spacing of
``3.0\verb+\+0xd''.
In fact, all entries will accept any string.  So entering a grid 
spacing of ``Bob'' will not result
in a editor error, but the planner will give an error (or strange
result).  Also, highlighting text in an entry has no use nor function.

\item With Polygon entry, if MaxVertices is set to 10, sometimes the user 
can enter a polygon with up to 10 vertices, other times up to 9.  This has 
to do with the way Tcl/Tk deals with lists of coordinates for a polygon.

\item The animate function will use the last path (default: {\tt push.path}) 
created.  Also, running the planner will write a temporary file called 
{\tt planner.tmp}.

\item Obstacle rotation always begins with 0 degrees, meaning if you rotate an 
obstacle, close the rotation window, and decide to rotate the object
again, the CURRENT position will be considered 0 degrees.

\item When saving both a problem and its path, the extension of the problem 
name is ignored when writing to the path.  For example, a problem
called ``test.problem'' would have an associated path name of
``test.path''.  A problem called ``test.fred'' would also have an
associated path name of ``test.path'', so different problem names
should not only differ by filename extension.
\end{itemize}

\subsection{Global variables}

Colors used in the object editor are defined in {\tt editcolors.tcl}.
The world editor uses the following global variables, specified in
{\tt globals.tcl}.

\begin{description}
\item[ColorFile]  The editor was designed to be used on a color monitor.
If you must use a monochrome monitor, however, set this variable to
{\tt editbw.tcl} instead of {\tt editcolors.tcl}.  This will make the editor
easier to use.
\item[PMConversion]  The conversion factor from screen pixels to ``virtual
millimeters,'' where ``millimeters'' = pixels/PMConversion.  
(Note that the unit is millimeters for use with the Adept.  Otherwise
the scale does not matter.)
\item[XMAX]  Default size of the world in the $x$ direction, in millimeters.
\item[YMAX]  Default size of the world in the $y$ direction, in millimeters.
\item[MaxObstacles]  Maximum number of obstacles that the editor will
allow to be placed in the world.
\item[MaxVertices]  Maximum number of vertices that an obstacle can have.
\end{description}

\section{The push planner}

The push planner {\tt pplanner} is made from {\tt pplanner.c}.
It is normally invoked by the world editor, but can also be
invoked from the command line by {\tt pplanner [-adept]} {\em $<$infile$>$},
where {\em infile\/} is the problem specification file.  If no
file is specified, the planner defaults to the file {\tt push.problem}.
If the {\tt -adept} flag is given, the planner will write a V+ pushing
program for an Adept robot.

The planner is a simple best-first search as described in the paper.
In this implementation, we only consider collisions between the 
object and the obstacles; we ignore collisions between the pushing
robot and the obstacles.  The robot is essentially assumed to be
an infinitely thin pushing surface.

\subsection{Global variables}

The planner uses the following global variables, defined in 
{\tt pglobals.h}.

\begin{description}
\item[INFILE] Problem file used by planner if no other file is specified (only relevant when running
planner from the command line).
\item[CONTROLOUT] Name of the control file written by the planner.
\item[PATHOUT] Name of the path file written by the planner (must be the same as editor global 
PathOut).
\item[MAXOBSTACLES] Maximum number of obstacles allowed in the world (must not be less than editor 
global MaxObstacles).
\item[MAXVERTICES] Maximum number of vertices allowed in an obstacle (must not be less than editor 
global MaxVertices).
\item[MAXCONTROLS] Maximum number of different velocities one pushing edge can be associated with.
\item[MAXEDGES] Maximum number of edges an object can have.
\item[MAXCOST] Upper limit on allowable cost of a pushing plan.
\item[ROBOTPOLY] Maximum number of polygons in the object representation.
\item[MAXNODES] Maximum number of search nodes.  This value should be 
as large as possible (preferably 1 million or more) subject to memory
limitations.
\item[XCELLS]  The maximum number of cells along the $x$-axis of the 
configuration space grid.  The size of each cell in the $x$ direction,
given in the world editor, should be sufficiently large that XCELLS cells
covers XMAX, the $x$ size of the world.
\item[YCELLS]  The maximum number of cells along the $y$-axis of the
configuration space grid.   The size of each cell in the $y$ direction,
given in the world editor, should be sufficiently large that YCELLS cells
covers YMAX, the $y$ size of the world.
\item[THETACELLS]  The maximum number of cells along the $\theta$-axis of the
configuration space grid.   The size of each cell in the $\theta$ direction,
given in the world editor, should be sufficiently large that THETACELLS cells
covers $2\pi$.
\item[TOOLOFFSET]  For use with Adept files.  Distance between the center of 
the gripper and the plane of the pusher.
\item[BACKUP]  For use with Adept files.  Distance to place the pusher
behind the object when lowering/raising the pusher.
\end{description}

\section{Running plans on an Adept robot}
The planner produces Adept programs by copying {\tt prog.header},
then writing the plan, and finally closing the program by copying
{\tt prog.trailer}.  The final file is {\tt push.pg}.

In our lab, we execute pushing plans with an Adept robot by doing the
following:
\begin{enumerate}
\item Create a pushing problem, solve it, and write the Adept file.
\item Print out a postscript version of the world and place it at
a fixed position in the Adept's workspace.  Place the object at the
start position on the paper world.
\item Transfer the file {\tt push.pg} to an Adept directory with the
file {\tt zero.pg}.  {\tt zero.pg} handles the case where joint 4
(gripper orientation) of the Adept reaches a joint limit during execution.
\item Load the two files into memory and run the program by 
{\tt execute pushing}.  
\end{enumerate}

To reproduce this setup in your lab, modify the file {\tt prog.header}
so that the BASE definition specifies the origin of your pushing world.
In the file {\tt pglobals.h}, set TOOLOFFSET to whatever is appropriate
for your pusher.  {\tt prog.header} and {\tt prog.trailer} assume that
the pusher is sitting at the location {\tt pusher}, and the Adept picks
up the pusher at the beginning of the plan and places it back at the
same position at the end.  Modify these two files according to your 
setup.

The dimensions of the printed world will be XMAX by YMAX in millimeters.

\end{document}
